 

\href{https://travis-ci.org/AFNetworking/AFNetworking}{\tt }

A\+F\+Networking is a delightful networking library for i\+OS and Mac OS X. It\textquotesingle{}s built on top of the \href{http://developer.apple.com/library/mac/#documentation/Cocoa/Conceptual/URLLoadingSystem/URLLoadingSystem.html}{\tt Foundation U\+RL Loading System}, extending the powerful high-\/level networking abstractions built into Cocoa. It has a modular architecture with well-\/designed, feature-\/rich A\+P\+Is that are a joy to use.

Perhaps the most important feature of all, however, is the amazing community of developers who use and contribute to A\+F\+Networking every day. A\+F\+Networking powers some of the most popular and critically-\/acclaimed apps on the i\+Phone, i\+Pad, and Mac.

Choose A\+F\+Networking for your next project, or migrate over your existing projects—you\textquotesingle{}ll be happy you did!

\subsection*{How To Get Started}


\begin{DoxyItemize}
\item \href{https://github.com/AFNetworking/AFNetworking/archive/master.zip}{\tt Download A\+F\+Networking} and try out the included Mac and i\+Phone example apps
\item Read the \href{https://github.com/AFNetworking/AFNetworking/wiki/Getting-Started-with-AFNetworking}{\tt \char`\"{}\+Getting Started\char`\"{} guide}, \href{https://github.com/AFNetworking/AFNetworking/wiki/AFNetworking-FAQ}{\tt F\+AQ}, or \href{https://github.com/AFNetworking/AFNetworking/wiki}{\tt other articles on the Wiki}
\item Check out the \href{http://cocoadocs.org/docsets/AFNetworking/}{\tt documentation} for a comprehensive look at all of the A\+P\+Is available in A\+F\+Networking
\item Read the \href{https://github.com/AFNetworking/AFNetworking/wiki/AFNetworking-2.0-Migration-Guide}{\tt A\+F\+Networking 2.\+0 Migration Guide} for an overview of the architectural changes from 1.\+0.
\end{DoxyItemize}

\subsection*{Communication}


\begin{DoxyItemize}
\item If you {\bfseries need help}, use \href{http://stackoverflow.com/questions/tagged/afnetworking}{\tt Stack Overflow}. (Tag \textquotesingle{}afnetworking\textquotesingle{})
\item If you\textquotesingle{}d like to {\bfseries ask a general question}, use \href{http://stackoverflow.com/questions/tagged/afnetworking}{\tt Stack Overflow}.
\item If you {\bfseries found a bug}, {\itshape and can provide steps to reliably reproduce it}, open an issue.
\item If you {\bfseries have a feature request}, open an issue.
\item If you {\bfseries want to contribute}, submit a pull request.
\end{DoxyItemize}

\subsubsection*{Installation with Cocoa\+Pods}

\href{https://cocoapods.org/}{\tt Cocoa\+Pods} is a dependency manager for Objective-\/C, which automates and simplifies the process of using 3rd-\/party libraries like A\+F\+Networking in your projects. See the \href{https://github.com/AFNetworking/AFNetworking/wiki/Getting-Started-with-AFNetworking}{\tt \char`\"{}\+Getting Started\char`\"{} guide for more information}.

\paragraph*{Podfile}


\begin{DoxyCode}
platform :ios, '7.0'
pod "AFNetworking", "~> 2.0"
\end{DoxyCode}


\subsection*{Requirements}

\tabulinesep=1mm
\begin{longtabu} spread 0pt [c]{*{4}{|X[-1]}|}
\hline
\rowcolor{\tableheadbgcolor}\textbf{ A\+F\+Networking Version  }&\textbf{ Minimum i\+OS Target  }&\textbf{ Minimum OS X Target  }&\textbf{ Notes   }\\\cline{1-4}
\endfirsthead
\hline
\endfoot
\hline
\rowcolor{\tableheadbgcolor}\textbf{ A\+F\+Networking Version  }&\textbf{ Minimum i\+OS Target  }&\textbf{ Minimum OS X Target  }&\textbf{ Notes   }\\\cline{1-4}
\endhead
2.\+x  &i\+OS 6  &OS X 10.\+8  &Xcode 5 is required. {\ttfamily N\+S\+U\+R\+L\+Session} subspec requires i\+OS 7 or OS X 10.\+9.   \\\cline{1-4}
\href{https://github.com/AFNetworking/AFNetworking/tree/1.x}{\tt 1.\+x}  &i\+OS 5  &Mac OS X 10.\+7  &\\\cline{1-4}
\href{https://github.com/AFNetworking/AFNetworking/tree/0.10.x}{\tt 0.\+10.\+x}  &i\+OS 4  &Mac OS X 10.\+6  &\\\cline{1-4}
\end{longtabu}


(OS X projects must support \href{https://developer.apple.com/library/mac/#documentation/Cocoa/Conceptual/ObjCRuntimeGuide/Articles/ocrtVersionsPlatforms.html}{\tt 64-\/bit with modern Cocoa runtime}).

\begin{quote}
Programming in Swift? Try \href{https://github.com/Alamofire/Alamofire}{\tt Alamofire} for a more conventional set of A\+P\+Is. \end{quote}


\subsection*{Architecture}

\subsubsection*{N\+S\+U\+R\+L\+Connection}


\begin{DoxyItemize}
\item {\ttfamily \mbox{\hyperlink{interface_a_f_u_r_l_connection_operation}{A\+F\+U\+R\+L\+Connection\+Operation}}}
\item {\ttfamily \mbox{\hyperlink{interface_a_f_h_t_t_p_request_operation}{A\+F\+H\+T\+T\+P\+Request\+Operation}}}
\item {\ttfamily \mbox{\hyperlink{interface_a_f_h_t_t_p_request_operation_manager}{A\+F\+H\+T\+T\+P\+Request\+Operation\+Manager}}}
\end{DoxyItemize}

\subsubsection*{N\+S\+U\+R\+L\+Session \+\_\+(i\+OS 7 / Mac OS X 10.\+9)\+\_\+}


\begin{DoxyItemize}
\item {\ttfamily A\+F\+U\+R\+L\+Session\+Manager}
\item {\ttfamily A\+F\+H\+T\+T\+P\+Session\+Manager}
\end{DoxyItemize}

\subsubsection*{Serialization}


\begin{DoxyItemize}
\item {\ttfamily $<$\mbox{\hyperlink{class_a_f_u_r_l_request_serialization-p}{A\+F\+U\+R\+L\+Request\+Serialization}}$>$}
\begin{DoxyItemize}
\item {\ttfamily \mbox{\hyperlink{interface_a_f_h_t_t_p_request_serializer}{A\+F\+H\+T\+T\+P\+Request\+Serializer}}}
\item {\ttfamily \mbox{\hyperlink{interface_a_f_j_s_o_n_request_serializer}{A\+F\+J\+S\+O\+N\+Request\+Serializer}}}
\item {\ttfamily \mbox{\hyperlink{interface_a_f_property_list_request_serializer}{A\+F\+Property\+List\+Request\+Serializer}}}
\end{DoxyItemize}
\item {\ttfamily $<$\mbox{\hyperlink{class_a_f_u_r_l_response_serialization-p}{A\+F\+U\+R\+L\+Response\+Serialization}}$>$}
\begin{DoxyItemize}
\item {\ttfamily \mbox{\hyperlink{interface_a_f_h_t_t_p_response_serializer}{A\+F\+H\+T\+T\+P\+Response\+Serializer}}}
\item {\ttfamily \mbox{\hyperlink{interface_a_f_j_s_o_n_response_serializer}{A\+F\+J\+S\+O\+N\+Response\+Serializer}}}
\item {\ttfamily \mbox{\hyperlink{interface_a_f_x_m_l_parser_response_serializer}{A\+F\+X\+M\+L\+Parser\+Response\+Serializer}}}
\item {\ttfamily A\+F\+X\+M\+L\+Document\+Response\+Serializer} \+\_\+(\+Mac O\+S X)\+\_\+
\item {\ttfamily \mbox{\hyperlink{interface_a_f_property_list_response_serializer}{A\+F\+Property\+List\+Response\+Serializer}}}
\item {\ttfamily \mbox{\hyperlink{interface_a_f_image_response_serializer}{A\+F\+Image\+Response\+Serializer}}}
\item {\ttfamily \mbox{\hyperlink{interface_a_f_compound_response_serializer}{A\+F\+Compound\+Response\+Serializer}}}
\end{DoxyItemize}
\end{DoxyItemize}

\subsubsection*{Additional Functionality}


\begin{DoxyItemize}
\item {\ttfamily \mbox{\hyperlink{interface_a_f_security_policy}{A\+F\+Security\+Policy}}}
\item {\ttfamily \mbox{\hyperlink{interface_a_f_network_reachability_manager}{A\+F\+Network\+Reachability\+Manager}}}
\end{DoxyItemize}

\subsection*{Usage}

\subsubsection*{H\+T\+TP Request Operation Manager}

{\ttfamily \mbox{\hyperlink{interface_a_f_h_t_t_p_request_operation_manager}{A\+F\+H\+T\+T\+P\+Request\+Operation\+Manager}}} encapsulates the common patterns of communicating with a web application over H\+T\+TP, including request creation, response serialization, network reachability monitoring, and security, as well as request operation management.

\paragraph*{{\ttfamily G\+ET} Request}


\begin{DoxyCode}
AFHTTPRequestOperationManager *manager = [AFHTTPRequestOperationManager manager];
[manager GET:@"http://example.com/resources.json" parameters:nil success:^(AFHTTPRequestOperation
       *operation, id responseObject) \{
    NSLog(@"JSON: %@", responseObject);
\} failure:^(AFHTTPRequestOperation *operation, NSError *error) \{
    NSLog(@"Error: %@", error);
\}];
\end{DoxyCode}


\paragraph*{{\ttfamily P\+O\+ST} U\+R\+L-\/\+Form-\/\+Encoded Request}


\begin{DoxyCode}
AFHTTPRequestOperationManager *manager = [AFHTTPRequestOperationManager manager];
NSDictionary *parameters =@"foo": @"bar"\};
[manager POST:@"http://example.com/resources.json" parameters:parameters success:^(AFHTTPRequestOperation
       *operation, id responseObject) \{
    NSLog(@"JSON: %@", responseObject);
\} failure:^(AFHTTPRequestOperation *operation, NSError *error) \{
    NSLog(@"Error: %@", error);
\}];
\end{DoxyCode}


\paragraph*{{\ttfamily P\+O\+ST} Multi-\/\+Part Request}


\begin{DoxyCode}
AFHTTPRequestOperationManager *manager = [AFHTTPRequestOperationManager manager];
NSDictionary *parameters =@"foo": @"bar"\};
NSURL *filePath = [NSURL fileURLWithPath:@"file://path/to/image.png"];
[manager POST:@"http://example.com/resources.json" parameters:parameters
       constructingBodyWithBlock:^(id<AFMultipartFormData> formData) \{
    [formData appendPartWithFileURL:filePath name:@"image" error:nil];
\} success:^(AFHTTPRequestOperation *operation, id responseObject) \{
    NSLog(@"Success: %@", responseObject);
\} failure:^(AFHTTPRequestOperation *operation, NSError *error) \{
    NSLog(@"Error: %@", error);
\}];
\end{DoxyCode}
 



\subsubsection*{A\+F\+U\+R\+L\+Session\+Manager}

{\ttfamily A\+F\+U\+R\+L\+Session\+Manager} creates and manages an {\ttfamily N\+S\+U\+R\+L\+Session} object based on a specified {\ttfamily N\+S\+U\+R\+L\+Session\+Configuration} object, which conforms to {\ttfamily $<$N\+S\+U\+R\+L\+Session\+Task\+Delegate$>$}, {\ttfamily $<$N\+S\+U\+R\+L\+Session\+Data\+Delegate$>$}, {\ttfamily $<$N\+S\+U\+R\+L\+Session\+Download\+Delegate$>$}, and {\ttfamily $<$N\+S\+U\+R\+L\+Session\+Delegate$>$}.

\paragraph*{Creating a Download Task}


\begin{DoxyCode}
NSURLSessionConfiguration *configuration = [NSURLSessionConfiguration defaultSessionConfiguration];
AFURLSessionManager *manager = [[AFURLSessionManager alloc] initWithSessionConfiguration:configuration];

NSURL *URL = [NSURL URLWithString:@"http://example.com/download.zip"];
NSURLRequest *request = [NSURLRequest requestWithURL:URL];

NSURLSessionDownloadTask *downloadTask = [manager downloadTaskWithRequest:request progress:nil
       destination:^NSURL *(NSURL *targetPath, NSURLResponse *response) \{
    NSURL *documentsDirectoryURL = [[NSFileManager defaultManager] URLForDirectory:NSDocumentDirectory
       inDomain:NSUserDomainMask appropriateForURL:nil create:NO error:nil];
    return [documentsDirectoryURL URLByAppendingPathComponent:[response suggestedFilename]];
\} completionHandler:^(NSURLResponse *response, NSURL *filePath, NSError *error) \{
    NSLog(@"File downloaded to: %@", filePath);
\}];
[downloadTask resume];
\end{DoxyCode}


\paragraph*{Creating an Upload Task}


\begin{DoxyCode}
NSURLSessionConfiguration *configuration = [NSURLSessionConfiguration defaultSessionConfiguration];
AFURLSessionManager *manager = [[AFURLSessionManager alloc] initWithSessionConfiguration:configuration];

NSURL *URL = [NSURL URLWithString:@"http://example.com/upload"];
NSURLRequest *request = [NSURLRequest requestWithURL:URL];

NSURL *filePath = [NSURL fileURLWithPath:@"file://path/to/image.png"];
NSURLSessionUploadTask *uploadTask = [manager uploadTaskWithRequest:request fromFile:filePath progress:nil
       completionHandler:^(NSURLResponse *response, id responseObject, NSError *error) \{
    if (error) \{
        NSLog(@"Error: %@", error);
    \} else \{
        NSLog(@"Success: %@ %@", response, responseObject);
    \}
\}];
[uploadTask resume];
\end{DoxyCode}


\paragraph*{Creating an Upload Task for a Multi-\/\+Part Request, with Progress}


\begin{DoxyCode}
NSMutableURLRequest *request = [[AFHTTPRequestSerializer serializer] multipartFormRequestWithMethod:@"POST"
       URLString:@"http://example.com/upload" parameters:nil constructingBodyWithBlock:^(id<AFMultipartFormData>
       formData) \{
        [formData appendPartWithFileURL:[NSURL fileURLWithPath:@"file://path/to/image.jpg"] name:@"file"
       fileName:@"filename.jpg" mimeType:@"image/jpeg" error:nil];
    \} error:nil];

AFURLSessionManager *manager = [[AFURLSessionManager alloc]
       initWithSessionConfiguration:[NSURLSessionConfiguration defaultSessionConfiguration]];
NSProgress *progress = nil;

NSURLSessionUploadTask *uploadTask = [manager uploadTaskWithStreamedRequest:request progress:&progress
       completionHandler:^(NSURLResponse *response, id responseObject, NSError *error) \{
    if (error) \{
        NSLog(@"Error: %@", error);
    \} else \{
        NSLog(@"%@ %@", response, responseObject);
    \}
\}];

[uploadTask resume];
\end{DoxyCode}


\paragraph*{Creating a Data Task}


\begin{DoxyCode}
NSURLSessionConfiguration *configuration = [NSURLSessionConfiguration defaultSessionConfiguration];
AFURLSessionManager *manager = [[AFURLSessionManager alloc] initWithSessionConfiguration:configuration];

NSURL *URL = [NSURL URLWithString:@"http://example.com/upload"];
NSURLRequest *request = [NSURLRequest requestWithURL:URL];

NSURLSessionDataTask *dataTask = [manager dataTaskWithRequest:request completionHandler:^(NSURLResponse
       *response, id responseObject, NSError *error) \{
    if (error) \{
        NSLog(@"Error: %@", error);
    \} else \{
        NSLog(@"%@ %@", response, responseObject);
    \}
\}];
[dataTask resume];
\end{DoxyCode}
 



\subsubsection*{Request Serialization}

Request serializers create requests from U\+RL strings, encoding parameters as either a query string or H\+T\+TP body.


\begin{DoxyCode}
NSString *URLString = @"http://example.com";
NSDictionary *parameters =@"foo": @"bar", @"baz": @[@1, @2, @3]\};
\end{DoxyCode}


\paragraph*{Query String Parameter Encoding}


\begin{DoxyCode}
[[AFHTTPRequestSerializer serializer] requestWithMethod:@"GET" URLString:URLString parameters:parameters
       error:nil];
\end{DoxyCode}
 \begin{DoxyVerb}GET http://example.com?foo=bar&baz[]=1&baz[]=2&baz[]=3
\end{DoxyVerb}


\paragraph*{U\+RL Form Parameter Encoding}


\begin{DoxyCode}
[[AFHTTPRequestSerializer serializer] requestWithMethod:@"POST" URLString:URLString parameters:parameters];
\end{DoxyCode}
 \begin{DoxyVerb}POST http://example.com/
Content-Type: application/x-www-form-urlencoded

foo=bar&baz[]=1&baz[]=2&baz[]=3
\end{DoxyVerb}


\paragraph*{J\+S\+ON Parameter Encoding}


\begin{DoxyCode}
[[AFJSONRequestSerializer serializer] requestWithMethod:@"POST" URLString:URLString parameters:parameters];
\end{DoxyCode}
 \begin{DoxyVerb}POST http://example.com/
Content-Type: application/json

{"foo": "bar", "baz": [1,2,3]}
\end{DoxyVerb}






\subsubsection*{Network \mbox{\hyperlink{interface_reachability}{Reachability}} Manager}

{\ttfamily \mbox{\hyperlink{interface_a_f_network_reachability_manager}{A\+F\+Network\+Reachability\+Manager}}} monitors the reachability of domains, and addresses for both W\+W\+AN and Wi\+Fi network interfaces.


\begin{DoxyItemize}
\item Do not use \mbox{\hyperlink{interface_reachability}{Reachability}} to determine if the original request should be sent.
\begin{DoxyItemize}
\item You should try to send it.
\end{DoxyItemize}
\item You can use \mbox{\hyperlink{interface_reachability}{Reachability}} to determine when a request should be automatically retried.
\begin{DoxyItemize}
\item Although it may still fail, a \mbox{\hyperlink{interface_reachability}{Reachability}} notification that the connectivity is available is a good time to retry something.
\end{DoxyItemize}
\item Network reachability is a useful tool for determining why a request might have failed.
\begin{DoxyItemize}
\item After a network request has failed, telling the user they\textquotesingle{}re offline is better than giving them a more technical but accurate error, such as \char`\"{}request timed out.\char`\"{}
\end{DoxyItemize}
\end{DoxyItemize}

See also \href{https://developer.apple.com/videos/play/wwdc2012-706/}{\tt W\+W\+DC 2012 session 706, \char`\"{}\+Networking Best Practices.\char`\"{}}.

\paragraph*{Shared Network \mbox{\hyperlink{interface_reachability}{Reachability}}}


\begin{DoxyCode}
[[AFNetworkReachabilityManager sharedManager]
       setReachabilityStatusChangeBlock:^(AFNetworkReachabilityStatus status) \{
    NSLog(@"Reachability: %@", AFStringFromNetworkReachabilityStatus(status));
\}];

[[AFNetworkReachabilityManager sharedManager] startMonitoring];
\end{DoxyCode}


\paragraph*{H\+T\+TP Manager \mbox{\hyperlink{interface_reachability}{Reachability}}}


\begin{DoxyCode}
NSURL *baseURL = [NSURL URLWithString:@"http://example.com/"];
AFHTTPRequestOperationManager *manager = [[AFHTTPRequestOperationManager alloc] initWithBaseURL:baseURL];

NSOperationQueue *operationQueue = manager.operationQueue;
[manager.reachabilityManager setReachabilityStatusChangeBlock:^(AFNetworkReachabilityStatus status) \{
    switch (status) \{
        case AFNetworkReachabilityStatusReachableViaWWAN:
        case AFNetworkReachabilityStatusReachableViaWiFi:
            [operationQueue setSuspended:NO];
            break;
        case AFNetworkReachabilityStatusNotReachable:
        default:
            [operationQueue setSuspended:YES];
            break;
    \}
\}];

[manager.reachabilityManager startMonitoring];
\end{DoxyCode}
 



\subsubsection*{Security Policy}

{\ttfamily \mbox{\hyperlink{interface_a_f_security_policy}{A\+F\+Security\+Policy}}} evaluates server trust against pinned X.\+509 certificates and public keys over secure connections.

Adding pinned S\+SL certificates to your app helps prevent man-\/in-\/the-\/middle attacks and other vulnerabilities. Applications dealing with sensitive customer data or financial information are strongly encouraged to route all communication over an H\+T\+T\+PS connection with S\+SL pinning configured and enabled.

\paragraph*{Allowing Invalid S\+SL Certificates}


\begin{DoxyCode}
AFHTTPRequestOperationManager *manager = [AFHTTPRequestOperationManager manager];
manager.securityPolicy.allowInvalidCertificates = YES; // not recommended for production
\end{DoxyCode}
 



\subsubsection*{\mbox{\hyperlink{interface_a_f_h_t_t_p_request_operation}{A\+F\+H\+T\+T\+P\+Request\+Operation}}}

{\ttfamily \mbox{\hyperlink{interface_a_f_h_t_t_p_request_operation}{A\+F\+H\+T\+T\+P\+Request\+Operation}}} is a subclass of {\ttfamily \mbox{\hyperlink{interface_a_f_u_r_l_connection_operation}{A\+F\+U\+R\+L\+Connection\+Operation}}} for requests using the H\+T\+TP or H\+T\+T\+PS protocols. It encapsulates the concept of acceptable status codes and content types, which determine the success or failure of a request.

Although {\ttfamily \mbox{\hyperlink{interface_a_f_h_t_t_p_request_operation_manager}{A\+F\+H\+T\+T\+P\+Request\+Operation\+Manager}}} is usually the best way to go about making requests, {\ttfamily \mbox{\hyperlink{interface_a_f_h_t_t_p_request_operation}{A\+F\+H\+T\+T\+P\+Request\+Operation}}} can be used by itself.

\paragraph*{{\ttfamily G\+ET} with {\ttfamily \mbox{\hyperlink{interface_a_f_h_t_t_p_request_operation}{A\+F\+H\+T\+T\+P\+Request\+Operation}}}}


\begin{DoxyCode}
NSURL *URL = [NSURL URLWithString:@"http://example.com/resources/123.json"];
NSURLRequest *request = [NSURLRequest requestWithURL:URL];
AFHTTPRequestOperation *op = [[AFHTTPRequestOperation alloc] initWithRequest:request];
op.responseSerializer = [AFJSONResponseSerializer serializer];
[op setCompletionBlockWithSuccess:^(AFHTTPRequestOperation *operation, id responseObject) \{
    NSLog(@"JSON: %@", responseObject);
\} failure:^(AFHTTPRequestOperation *operation, NSError *error) \{
    NSLog(@"Error: %@", error);
\}];
[[NSOperationQueue mainQueue] addOperation:op];
\end{DoxyCode}


\paragraph*{Batch of Operations}


\begin{DoxyCode}
NSMutableArray *mutableOperations = [NSMutableArray array];
for (NSURL *fileURL in filesToUpload) \{
    NSURLRequest *request = [[AFHTTPRequestSerializer serializer] multipartFormRequestWithMethod:@"POST"
       URLString:@"http://example.com/upload" parameters:nil constructingBodyWithBlock:^(id<AFMultipartFormData>
       formData) \{
        [formData appendPartWithFileURL:fileURL name:@"images[]" error:nil];
    \}];

    AFHTTPRequestOperation *operation = [[AFHTTPRequestOperation alloc] initWithRequest:request];

    [mutableOperations addObject:operation];
\}

NSArray *operations = [AFURLConnectionOperation batchOfRequestOperations:@[...] progressBlock:^(NSUInteger
       numberOfFinishedOperations, NSUInteger totalNumberOfOperations) \{
    NSLog(@"%lu of %lu complete", numberOfFinishedOperations, totalNumberOfOperations);
\} completionBlock:^(NSArray *operations) \{
    NSLog(@"All operations in batch complete");
\}];
[[NSOperationQueue mainQueue] addOperations:operations waitUntilFinished:NO];
\end{DoxyCode}


\subsection*{Unit Tests}

A\+F\+Networking includes a suite of unit tests within the Tests subdirectory. In order to run the unit tests, you must install the testing dependencies via \href{https://cocoapods.org/}{\tt Cocoa\+Pods}\+: \begin{DoxyVerb}$ cd Tests
$ pod install
\end{DoxyVerb}


Once testing dependencies are installed, you can execute the test suite via the \textquotesingle{}i\+OS Tests\textquotesingle{} and \textquotesingle{}OS X Tests\textquotesingle{} schemes within Xcode.

\subsubsection*{Running Tests from the Command Line}

Tests can also be run from the command line or within a continuous integration environment. The \href{https://github.com/supermarin/xcpretty}{\tt {\ttfamily xcpretty}} utility needs to be installed before running the tests from the command line\+: \begin{DoxyVerb}$ gem install xcpretty
\end{DoxyVerb}


Once {\ttfamily xcpretty} is installed, you can execute the suite via {\ttfamily rake test}.

\subsection*{Credits}

A\+F\+Networking is owned and maintained by the \href{http://alamofire.org}{\tt Alamofire Software Foundation}.

A\+F\+Networking was originally created by \href{https://github.com/sco/}{\tt Scott Raymond} and \href{https://github.com/mattt/}{\tt Mattt Thompson} in the development of \href{https://en.wikipedia.org/wiki/Gowalla}{\tt Gowalla for i\+Phone}.

A\+F\+Networking\textquotesingle{}s logo was designed by \href{http://www.alandefibaugh.com/}{\tt Alan Defibaugh}.

And most of all, thanks to A\+F\+Networking\textquotesingle{}s \href{https://github.com/AFNetworking/AFNetworking/graphs/contributors}{\tt growing list of contributors}.

\subsubsection*{Security Disclosure}

If you believe you have identified a security vulnerability with A\+F\+Networking, you should report it as soon as possible via email to \href{mailto:security@alamofire.org}{\tt security@alamofire.\+org}. Please do not post it to a public issue tracker.

\subsection*{License}

A\+F\+Networking is released under the M\+IT license. See L\+I\+C\+E\+N\+SE for details. 